\documentclass[a4paper, 12pt]{article} % type of document, paper format,~~~~length of tab

% include Russian.
\usepackage[T2A]{fontenc}
\usepackage[utf8]{inputenc} 
\usepackage[english, russian]{babel}

\usepackage{ upgreek }
\usepackage[table,xcdraw]{xcolor}


\usepackage{indentfirst} %tabutation

\usepackage{amsmath, amsfonts, amssymb, amsthm, mathtools} %include math

\begin{titlepage}
\pagestyle{empty}
\title{Работа 1.1.4

Измерение интенсивности радиационнного фона}

\author{Устюжанина Мария Алексеевна}
\date{Сентябрь 2021}

\end{titlepage}


\begin{document}


\maketitle

\newpage

\setcounter{page}{1}

\section*{Аннотация}
\subsection*{Тема}
Измерение интенсивности радиационнного фона.

\subsection*{Цель работы}
Применение методов обработки экспериментальных данных для изучения закономерностей при измерении интенсивности радиоактивного фона.

\subsection*{Оборудование}
Cчетчик Гейгера-Мюллера(СТС-6), блок питания, компьютер с интерфейсом связи с счетчиком.

\subsection*{Измеряемые параметры}
Число зарегестрированных частиц.

\subsection*{Методы обработки полученных измерений}
Гистограммы, таблицы


\section*{Теоретические сведения}

Обнаружить космические лучи, составляющие значительную часть радиационного фона, и измерить их интенсивность можно по ионизации, которую они производят. Для этого используется счетчик Гейгера-Мюллера. Он представляет собой наполненный газом сосуд с двумя электродами. Существует несколько типов таких счетчиков.

В работе используется СТС-6. Данный счетчик представляет собой тонкостенный металлический цилиндр, наполненный газом (который является катодом), с тонкой нитью-анодом, натянутой вдоль оси. Для работы счетчика на электроны подают напряжение 400В.

Частицы космических лучей ионизируют газ и выбивают электроны. В результате взаимодействий между частицами образуется лавина электронов и ток через счетчик резко увеличивается. Во время разряда 

Число зарегистрированных частиц зависит от времени измерения, размеров счетчика, материала стенок, состава газа и давления в нем. Значительную часть зарегистрированных частиц составляет естественный радиоактивный фон.

Погрешности измерений потока частиц с помощью счетчика Гейгера-Мюллера малы по сравнению с изменением самого потока, и определяются в основном временем, в течение которого восстанавливаются нормальные условия после прохождения каждой частицы и срабатывания счетчика (временем разрешения).


\section*{Результаты измерений и обработка данных}

\begin{enumerate}

\item
Измеряем плотность потока космического излучения за 20 секунд. Обработанный компьютером результат представлен в таблице 1.

\item
Разбив результат из таблицы 1 в порядке их получения на группы по 2, что будет соответствовать произведению \(N_2=100\) измерений число частиц за 40 секунд. Результаты приведены в таблице 2.

\item
Данные для построения гистограмм распределения числа срабатываний счетчика за 10 секунд и за 40 секунд представлены в таблицах 3 и 4 соответственно.

\item
Гистограммы распределений среднего числа отсчетов за 10 и 40 секунд приведена на рисунке 1.

\item
Используя формул:

\[ \overline{n_1}=\frac{1}{N_1}\cdot\sum\limits_{i=1}^{N_1}n_i\]

\[ \overline{n_2}=\frac{1}{N_2}\cdot\sum\limits_{i=1}^{N_2}n_i\]

Найдем среднее число срабатываний счетчика за 10 и 40 секунд соответственно, взяв данные из таблицы 1:

\[ \overline{n_1}=\frac{4514}{400}=11,29\]

\[ \overline{n_2}=\frac{4374}{100}=43,74\]

\item
Вычислим среднеквадратичные ошибки отдельных измерений:

\[ \sigma_1=\sqrt{\frac{1}{N_1}\cdot\sum\limits_{i=1}^{N_1}(n_i-\overline{n_1})^2}=\sqrt{\frac{4545,52}{400}}\approx 3,37\]

\[ \sigma_2=\sqrt{\frac{1}{N_2}\cdot\sum\limits_{i=1}^{N_2}(n_i-\overline{n_2})^2}=\sqrt{\frac{4193,76}{100}}\approx 6,48 \]

\item
Проверим справедливость формул:
\[ \sigma_1\approx \sqrt{\overline{n_1}} \approx \sqrt{11,29}  \approx 3,36\]

\[ \sigma_2\approx \sqrt{\overline{n_2}} \approx \sqrt{43,74}  \approx 6,61\]

\item
Определим стандартную ошибку величины \( \overline{n_1}, \overline{n_2}\) и относительную ошибку нахождения \( \overline{n_1} \) и \( \overline{n_2}\) по формулам:

\[ \sigma_{\overline{n_1}} = \frac{\sigma_1}{\sqrt{N_1}} = \frac{3,37}{400} \approx 0,1685\]

\[ \sigma_{\overline{n_2}} = \frac{\sigma_2}{\sqrt{N_2}} = \frac{6,48}{100} \approx 0,648\]

\[ \varepsilon_{\overline{n_1}} = \frac{\sigma_{\overline{n_1}}}{\overline{n_1}} \cdot 100\% = \frac{0,1685}{11,29} \cdot 100 \% \approx1.49\% \]

\[ \varepsilon_{\overline{n_2}} = \frac{\sigma_{\overline{n_2}}}{\overline{n_2}} \cdot 100\% = \frac{0,648}{43,74} \cdot 100 \% \approx1.48\% \]

\item
Найдем относительную ошибку через равенство:

\[ \varepsilon_{\overline{n_1}} = \frac{100\%}{\sqrt{\overline{n_1} \cdot N_1}} = \frac{100\%}{\sqrt{11,29 \cdot 400}} \approx 1,49\% \]

\[ \varepsilon_{\overline{n_2}} = \frac{100\%}{\sqrt{\overline{n_2} \cdot N_2}} = \frac{100\%}{\sqrt{43,74 \cdot 100}} \approx 1.51\% \]

\item
Окончательный результат:

\[ n_{t=10c} = \overline{n_1} \pm \sigma_{\overline{n_1}} = 11,29 \pm 0,17\]

\[ n_{t=40c} = \overline{n_2} \pm \sigma_{\overline{n_2}} = 43,74 \pm 0,65\]

\item
Все результаты занесем в таблицу 5.

%\item
%Определим долю случаев, когда отклонения от среднего значения не превышают \( \sigma_1, 2\sigma_2 \ (\sigma_2, 2\sigma_2) \)(таблица 6).

\item
Сравним среднеквадратичные ошибки отдельных измерений для двух распределений \(overline{n_1} = 11,29; \ \sigma_1 = 3,37\) и \(overline{n_2} = 43,74; \ \sigma_2 = 6,48\). Легко заметить, что хотя абсолютное значение \(\sigma\) во втором распределении больше, чем в первом, относительная погрешность второго распределения меньше:

\[ \frac{\sigma_1}{\overline{n_1}} \cdot 100\% = \frac{3,37}{11,29} \approx 29,8 \% \]

\[ \frac{\sigma_2}{\overline{n_2}} \cdot 100\% = \frac{6,48}{43,74} \approx 14,8 \% \]

Это также следует из рисунка 1 и 2.

\end{enumerate}

\newpage

\* \textbf{Таблица 1. Число срабатываний счетчика за 20 секунд}
\begin{figure}[h]
\centering
\includegraphics[width=0.5\linewidth]{Tabl1}
\label{fig:mpr}
\end{figure}

\  

\* \textbf{Таблица 2. Число срабатываний счетчика за 40 секунд}

\begin{tabular}{lllllllllll}
\rowcolor[HTML]{9B9B9B} 
№ опыта                      & 1  & 2  & 3                       & 4                       & 5  & 6  & 7  & 8  & 9  & 10 \\
\cellcolor[HTML]{C0C0C0}0  & 41 & 50 & 48 & 45                      & 47 & 46 & 40 & 48 & 42 & 48 \\
\cellcolor[HTML]{C0C0C0}10 & 24 & 46 & 40& 53  & 48 & 40 & 49 & 47 & 35 & 45 \\ 
\cellcolor[HTML]{C0C0C0}20 & 45 & 52 & 47 & 50 & 43 & 39 & 40 & 45 & 49 & 38 \\ 
\cellcolor[HTML]{C0C0C0}30 & 45 & 65 & 35                      & 52                      & 45 & 52 & 27 & 53 & 49 & 50 \\
\cellcolor[HTML]{C0C0C0}40 & 45 & 58 & 50                      & 41                      & 58 & 27 & 49 & 42 & 33 & 38 \\
\cellcolor[HTML]{C0C0C0}50 & 45 & 47 & 50                      & 43                      & 40 & 48 & 46 & 51 & 37 & 35 \\
\cellcolor[HTML]{C0C0C0}60 & 48 & 49 & 54                      & 43                      & 48 & 54 & 37 & 43 & 37 & 46 \\
\cellcolor[HTML]{C0C0C0}70 & 48 & 44 & 42                      & 34                      & 43 & 44 & 40 & 42 & 38 & 42 \\
\cellcolor[HTML]{C0C0C0}80 & 38 & 38 & 36                      & 47                      & 56 & 47 & 53 & 38 & 35 & 39 \\
\cellcolor[HTML]{C0C0C0}90 & 54 & 58 & 50                      & 52                      & 33 & 46 & 32 & 53 & 51 & 45
\end{tabular}

\newpage

\* \textbf{Таблица 3.}
\begin{figure}[h]
\centering
\includegraphics[width=0.5\linewidth]{Tabl3}
\label{fig:mpr}
\end{figure}


\* \textbf{Таблица 4. Данные для гистограммы t = 40 секунд}

\begin{tabular}{llllllllllll}
\cellcolor[HTML]{EFEFEF}Число импульсов & 27   & 32   & 33   & 34   & 35   & 36   & 37   & 38   & 39   & 40   & 41   \\
\cellcolor[HTML]{EFEFEF}Число случаев   & 2    & 1    & 2    & 1    & 4    & 1    & 3    & 6    & 2    & 6    & 2    \\
\cellcolor[HTML]{EFEFEF}Доля случаев    & 0,02 & 0,01 & 0,02 & 0,01 & 0,04 & 0,01 & 0,03 & 0,06 & 0,02 & 0,06 & 0,02 \\
\rowcolor[HTML]{9B9B9B} 
Число импульсов                         & 42   & 43   & 44   & 45   & 46   & 47   & 48   & 49   & 50   & 51   & 52   \\
\rowcolor[HTML]{9B9B9B} 
Число случаев                           & 5    & 5    & 2    & 9    & 5    & 6    & 8    & 5    & 6    & 2    & 4    \\
\rowcolor[HTML]{9B9B9B} 
Доля случаев                            & 0,05 & 0,05 & 0,02 & 0,09 & 0,05 & 0,06 & 0,08 & 0,05 & 0,06 & 0,02 & 0,04 \\
\cellcolor[HTML]{EFEFEF}Число импульсов & 53   & 54   & 55   & 56   & 57   & 58   & 59   & 60   & 61   & 62   & 63   \\
\cellcolor[HTML]{EFEFEF}Число случаев   & 4    & 3    & 0    & 1    & 0    & 3    & 0    & 0    & 0    & 0    & 0    \\
\cellcolor[HTML]{EFEFEF}Доля случаев    & 0,04 & 0,03 & 0    & 0,01 & 0    & 0,03 & 0    & 0    & 0    & 0    & 0   
\end{tabular}

\ 


\* \textbf{Таблица 5. Ошибки и средние значения}

\begin{tabular}{lllll}
\rowcolor[HTML]{EFEFEF} 
                          & \(\overline{n}\)     & \( \sigma \) среднекв.   & \( \sigma \) пример.   & \( \sigma_{\overline{n}}; \sigma \)  cтанд.     \\
\cellcolor[HTML]{EFEFEF}1 & 11,29 & 3,37 & 3,36 & 0,1685 \\
\cellcolor[HTML]{EFEFEF}2 & 43,74 & 6,48 & 6,61 & 0,648 
\end{tabular}

\newpage

\* \textbf{Рисунок 1. Гистограмма за 10с}
\begin{figure}[h]
\centering
\includegraphics[width=0.6\linewidth]{Paint1}
\label{fig:mpr}
\end{figure}

\* \textbf{Рисунок 2. Гистограмма за 40с}
\begin{figure}[h]
\centering
\includegraphics[width=0.8\linewidth]{Paint2}
\label{fig:mpr}
\end{figure}


%\* \textbf{Таблица 6. Данные для гистограммы t = 40 секунд}


\end{document}



